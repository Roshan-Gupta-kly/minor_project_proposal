% include the page wise photos of plagarism summary document:
% \chapter{MATHEMATICAL DERIVATIONS}
% You can include the Derivation and mathematical part of your project that is not necessary to be included in the main report here. To provide reader with the complete mathematical background of your project, you can include the derivations, proofs, and detailed explanations of theorems or formulas used in your project. and then you can cross reference them in the main report.
% \section{Proof of Theorem 1:} Convex Optimization Problem can be solved using Gradient Descent Method.
% \section{Proof of Theorem 2:} Convergence of Gradient Descent Method for Convex Functions.
% \section{Gaussian Distribution} 
% The Gaussian distribution, also known as the normal distribution, is a continuous probability distribution characterized by its bell-shaped curve. It is defined by its mean ($\mu$) and standard deviation ($\sigma$). The probability density function (PDF) of a Gaussian distribution is given by:
% \[
% f(x) = \frac{1}{\sigma\sqrt{2\pi}} \exp\!\left(-\frac{(x-\mu)^2}{2\sigma^2}\right).
% \]



\chapter{PROJECT BUDGET}
The total budget required for the successful completion of the project is estimated to be between \textbf{NPR 8000 - NPR 11000}. This budget covers various aspects of the project, including hardware purchases, component materials, software services, and miscellaneous expenses.

\section{Bill of Materials (BOM)}
This section details all components that will be integrated into the final product/system. These are the materials that constitute the deliverable hardware (if any).

\begin{table}[H]
    \centering
    \small
    \caption{Project Budget}
    \begin{tabular}{p{1cm}p{5cm}ccc}
        \toprule
        \textbf{SN} & \textbf{Components} & \textbf{Qty} & \textbf{Unit Cost (NPR)} & \textbf{Total (NPR)} \\
        \midrule
        1  & ESP32 S3           & 2  & 1500 & 3000 \\
        2  & GSM800L            & 1  & 1000 & 1000 \\
        3  & GPS                & 1  & 800  & 800  \\
        4  & MPU6050            & 1  & 700  & 700  \\
        5  & HMC5883L           & 1  & 450  & 450  \\
        6  & Flame Sensor       & 1  & 320  & 320  \\
        7  & Crash Sensor       & 1  & 100  & 100  \\
        8  & SIM Card           & 1  & 100  & 100  \\
        9  & Bread Board        & 1  & 450  & 450  \\
        10 & Resistor           & 1  & 150  & 150  \\
        11 & Buzzer             & 1  & 100  & 100  \\
        12 & Wire               & 1  & 200  & 200  \\
        13 & Power Supply       & 1  & 400  & 400  \\
        \midrule
        \multicolumn{4}{r}{\textbf{Total}} & \textbf{7770} \\
        \bottomrule
    \end{tabular}
    \label{tab:project_budget}
\end{table}



% \section{Software and Cloud Services}
% This section covers all software tools, cloud computing resources, and subscription services required for the project.

% \begin{table}[H]
%     \centering
%     \small
%     \caption{Software and Cloud Services Cost}
%     \begin{tabular}{p{3.5cm}p{5cm}p{2.5cm}}
%         \toprule
%         \textbf{Service}                              & \textbf{Description}                            & \textbf{Cost} \\
%         \midrule
%         Open-Source Software                          & Python, NumPy, SciPy, TensorFlow, PyTorch, etc. & Free          \\
%         Cloud Computing                               & Google Colab Pro (GPU access for training)      & \$10/month    \\
%         Cloud Storage                                 & Google Drive (100GB)                            & \$2/month     \\
%         API Services                                  & Third-party API usage (if applicable)           & \$15          \\
%         IEEE Access                                   & Student subscription for research papers        & \$15          \\
%         \midrule
%         \multicolumn{2}{r}{\textbf{Total (3 months)}} & \textbf{\$66}                                                   \\
%         \bottomrule
%     \end{tabular}
%     \label{tab:software_services}
% \end{table}


% \section{Miscellaneous Expenses}
% This section includes other project-related costs such as documentation, printing, and administrative expenses.

% The complete budget estimation is shown in \cref{tab:cost_estimation_cloud}. The budget is designed to ensure that all required resources are available for the successful completion of the project.


% This expense breakdown in \cref{tab:cost_estimation_cloud} incorporates the first half of the project.

% \textbf{Note:} All tables in this document use the \texttt{booktabs} package for professional table formatting, as per IEEE style guidelines. Use \verb|\toprule|, \verb|\midrule|, and \verb|\bottomrule| instead of \verb|\hline| for better visual appearance.

% \begin{table}[H]
%     \centering
%     \small
%     \caption{Estimated Cost Breakdown}
%     \begin{tabular}{p{2.7cm}p{6.5cm}p{2.5cm}}
%         \toprule
%         \textbf{Category}                                 & \textbf{Item/Description}                                                                                             & \textbf{Estimated Cost}     \\
%         \midrule
%         Software                                          & Python, NumPy, SciPy, DEAP, SQLite, Matplotlib, Seaborn, Gym (for RL), and other required libraries (all open-source) & Free                        \\
%         Cloud Computing                                   & \textbf{Google Cloud Platform (GCP)}: \newline
%         - NVIDIA T4: \$0.35/hour \newline
%         - NVIDIA V100: \$2.48/hour \newline
%         - NVIDIA A100 (40GB): \$4.27/hour \newline
%         \textbf{Google Colab Pro}: \newline
%         - \$9.99/month for 100 compute units \newline
%         - Approx. 13 units/hour for A100 usage            & Estimated \$300-\$350                                                                                                                               \\
%         IEEE Access                                       & Annual subscription for accessing IEEE research articles and journals                                                 & \$15 (student subscription) \\
%         Miscellaneous                                     & Printing, documentation, report binding                                                                               & \$15-\$25                   \\
%         \midrule
%         \multicolumn{2}{l}{\textbf{Total Estimated Cost}} & \textbf{\$330-\$390}                                                                                                                                \\
%         \bottomrule
%     \end{tabular}
%     \label{tab:cost_estimation_cloud}
% \end{table}

\chapter{PROJECT TIMELINE}
The project will be executed over a period of approximately 6 months, divided into several key phases. Each phase includes specific tasks and milestones to ensure steady progress toward project completion.

\section{Gantt Chart}
This is the estimated timeline for the project, detailing the key phases and milestones. The project is structured into several stages, each with specific tasks and deliverables.

\begin{figure}
    \centering
    \includegraphics[width=\textwidth]{src/images/figures/gantt.jpg} % Replace with actual path to Gantt chart image
    \caption{Project Gantt Chart}
    \label{fig:gantt_chart}
\end{figure}

\chapter{FEASIBILITY}
\chapter{FEASIBILITY}

This project is proposed to be implemented using hardware components such as the \gls{esp32s3} microcontroller, \gls{acc}, \gls{gy}, \gls{ms}, \gls{fs}, \gls{cs}, ultrasonic sensor, \gls{gps} module, and \gls{gsm} module. These components are readily available in the market and are cost-effective, making the system economically feasible for real-world deployment.

The software tools used in this project are free and open-source. Programming of the \gls{esp32s3} is carried out using the Arduino \gls{ide} and \gls{espidf}. For data analysis and \gls{ml} model development, Python-based libraries such as NumPy and Pandas are used. These tools support sensor data processing, feature extraction, and model training during the learning phase of the project.

The dataset required for training and validating the prototype is planned to be obtained from open-source platforms such as Kaggle, along with real-time data collected from the sensors during testing. This combination helps improve model reliability and provides insight into real-world sensor behavior.

The project requires a \gls{sim} card and a \gls{gsm}-enabled mobile network to function effectively. The system is installed in a vehicle with a \gls{sim} card inserted into the \gls{gsm} module, enabling communication over the mobile network. This allows the \gls{esp32s3} to send alerts, \gls{sos} messages, and location information to predefined contacts or emergency services during accident or fire detection events.



% \chapter{PLAGARISM SUMMARY}
% Keep all the pages of the plagarism summary document provided by the university here. It can run for multiple pages.
