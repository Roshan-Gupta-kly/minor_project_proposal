% include the page wise photos of plagarism summary document:
\chapter{MATHEMATICAL DERIVATIONS}
You can include the Derivation and mathematical part of your project that is not necessary to be included in the main report here. To provide reader with the complete mathematical background of your project, you can include the derivations, proofs, and detailed explanations of theorems or formulas used in your project. and then you can cross reference them in the main report.
\section{Proof of Theorem 1:} Convex Optimization Problem can be solved using Gradient Descent Method.
\section{Proof of Theorem 2:} Convergence of Gradient Descent Method for Convex Functions.
\section{Gaussian Distribution} 
The Gaussian distribution, also known as the normal distribution, is a continuous probability distribution characterized by its bell-shaped curve. It is defined by its mean ($\mu$) and standard deviation ($\sigma$). The probability density function (PDF) of a Gaussian distribution is given by:
\[
f(x) = \frac{1}{\sigma\sqrt{2\pi}} \exp\!\left(-\frac{(x-\mu)^2}{2\sigma^2}\right).
\]



\chapter{PROJECT BUDGET}
The total budget required for the successful completion of the project is estimated to be between \$330 and \$390. This budget covers various aspects of the project, including hardware purchases, component materials, software services, and miscellaneous expenses.

\section{Development Equipment}
This section lists essential hardware and tools that must be purchased for development and testing. These are items not available in the lab or not part of the final product, but necessary for the project.

\begin{table}[H]
    \centering
    \small
    \caption{Development Equipment Cost}
    \begin{tabular}{p{3cm}p{4cm}ccc}
        \toprule
        \textbf{Item}                         & \textbf{Specification}      & \textbf{Qty} & \textbf{Unit Cost} & \textbf{Total} \\
        \midrule
        Development Board                     & Raspberry Pi 4 (4GB)        & 1            & \$55               & \$55           \\
        Testing Equipment                     & Digital Multimeter          & 1            & \$25               & \$25           \\
        Sensors                               & Temperature/Humidity Sensor & 2            & \$10               & \$20           \\
        \midrule
        \multicolumn{4}{r}{\textbf{Subtotal}} & \textbf{\$100}                                                                   \\
        \bottomrule
    \end{tabular}
    \label{tab:dev_equipment}
\end{table}

\textit{Note: Replace the example entries above with actual equipment needed for your project.}

\section{Bill of Materials (BOM)}
Keep this only if relevant to your project.
This section details all components that will be integrated into the final product/system. These are the materials that constitute the deliverable hardware (if any).

\begin{table}[H]
    \centering
    \small
    \caption{Bill of Materials}
    \begin{tabular}{p{3cm}p{4cm}ccc}
        \toprule
        \textbf{Component}                    & \textbf{Specification} & \textbf{Qty} & \textbf{Unit Cost} & \textbf{Total} \\
        \midrule
        Microcontroller                       & ESP32 DevKit           & 1            & \$10               & \$10           \\
        Sensor                                & DHT22 Sensor           & 2            & \$5                & \$10           \\
        Actuator                              & Servo Motor SG90       & 2            & \$3                & \$6            \\
        Power Supply                          & 5V/2A Adapter          & 1            & \$8                & \$8            \\
        PCB                                   & Custom PCB (100x100mm) & 1            & \$15               & \$15           \\
        Enclosure                             & ABS Plastic Box        & 1            & \$12               & \$12           \\
        \midrule
        \multicolumn{4}{r}{\textbf{Subtotal}} & \textbf{\$61}                                                               \\
        \bottomrule
    \end{tabular}
    \label{tab:bom}
\end{table}

\textit{Note: Replace the example entries above with actual components for your project.}

\section{Software and Cloud Services}
This section covers all software tools, cloud computing resources, and subscription services required for the project.

\begin{table}[H]
    \centering
    \small
    \caption{Software and Cloud Services Cost}
    \begin{tabular}{p{3.5cm}p{5cm}p{2.5cm}}
        \toprule
        \textbf{Service}                              & \textbf{Description}                            & \textbf{Cost} \\
        \midrule
        Open-Source Software                          & Python, NumPy, SciPy, TensorFlow, PyTorch, etc. & Free          \\
        Cloud Computing                               & Google Colab Pro (GPU access for training)      & \$10/month    \\
        Cloud Storage                                 & Google Drive (100GB)                            & \$2/month     \\
        API Services                                  & Third-party API usage (if applicable)           & \$15          \\
        IEEE Access                                   & Student subscription for research papers        & \$15          \\
        \midrule
        \multicolumn{2}{r}{\textbf{Total (3 months)}} & \textbf{\$66}                                                   \\
        \bottomrule
    \end{tabular}
    \label{tab:software_services}
\end{table}

\textit{Note: Adjust the entries above based on your actual software and service requirements.}

\section{Miscellaneous Expenses}
This section includes other project-related costs such as documentation, printing, and administrative expenses.

The complete budget estimation is shown in \cref{tab:cost_estimation_cloud}. The budget is designed to ensure that all required resources are available for the successful completion of the project.


This expense breakdown in \cref{tab:cost_estimation_cloud} incorporates the first half of the project.

\textbf{Note:} All tables in this document use the \texttt{booktabs} package for professional table formatting, as per IEEE style guidelines. Use \verb|\toprule|, \verb|\midrule|, and \verb|\bottomrule| instead of \verb|\hline| for better visual appearance.

\begin{table}[H]
    \centering
    \small
    \caption{Estimated Cost Breakdown}
    \begin{tabular}{p{2.7cm}p{6.5cm}p{2.5cm}}
        \toprule
        \textbf{Category}                                 & \textbf{Item/Description}                                                                                             & \textbf{Estimated Cost}     \\
        \midrule
        Software                                          & Python, NumPy, SciPy, DEAP, SQLite, Matplotlib, Seaborn, Gym (for RL), and other required libraries (all open-source) & Free                        \\
        Cloud Computing                                   & \textbf{Google Cloud Platform (GCP)}: \newline
        - NVIDIA T4: \$0.35/hour \newline
        - NVIDIA V100: \$2.48/hour \newline
        - NVIDIA A100 (40GB): \$4.27/hour \newline
        \textbf{Google Colab Pro}: \newline
        - \$9.99/month for 100 compute units \newline
        - Approx. 13 units/hour for A100 usage            & Estimated \$300-\$350                                                                                                                               \\
        IEEE Access                                       & Annual subscription for accessing IEEE research articles and journals                                                 & \$15 (student subscription) \\
        Miscellaneous                                     & Printing, documentation, report binding                                                                               & \$15-\$25                   \\
        \midrule
        \multicolumn{2}{l}{\textbf{Total Estimated Cost}} & \textbf{\$330-\$390}                                                                                                                                \\
        \bottomrule
    \end{tabular}
    \label{tab:cost_estimation_cloud}
\end{table}

\chapter{PROJECT TIMELINE}
The project will be executed over a period of approximately 6 months, divided into several key phases. Each phase includes specific tasks and milestones to ensure steady progress toward project completion.
\section{Gantt Chart}
gantt should clearly show the year month, week, and tasks with their durations and dependencies, with milestones also properly marked.
This is the estimated timeline for the project, detailing the key phases and milestones. The project is structured into several stages, each with specific tasks and deliverables.

\textbf{Gantt Chart Placeholder here:}
\begin{figure}
    \centering
    \includegraphics[width=\textwidth]{src/images/figures/gantt.png} % Replace with actual path to Gantt chart image
    \caption{Project Gantt Chart}
    \label{fig:gantt_chart}
\end{figure}


\chapter{PLAGARISM summary}
Keep all the pages of the plagarism summary document provided by the university here. It can run for multiple pages.
