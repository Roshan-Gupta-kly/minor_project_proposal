\chapter{Introduction}

\noindent\textbf{CHAPTER GUIDE:} “EDR (Event Data Recorder) Based Intelligent Vehicle Accident Prediction with Multisensor Fusion and Edge Summarization” is a project that focuses on pre-event and post-event analysis and predict the accident and log it. The MultiSensor Fusion gives more accurate data and we can use that data to feed into a real time frame. The edge processing eliminates the unnecessary flagged data that we do not need while analyzing and alerting the system. The traditional project lacked the fusion of two technologies in a single EDR, which just doesn’t make this a logger rather make it an intelligent predicting logger further explained in block diagram of methodology.


\section{Background}

% Test abbreviations and symbols
Conventional vehicle EDR lacked the intelligent decision making capabilities. The project highlights an intelligent accident intelligent Vehicle Accident Prediction with Multisensor Fusion and Edge Summarization. The accuracy and timely emergency response results in a safety driven concern for the vehicles. The alert system that is classified accordingly makes the use case of this project more practical and more reliable compared to the traditional EDR. Multisensor fusion and edge summarization can improve overall reliability which is more practical in real-world deployment. 

\section{Motivation}
This project came into idea when there was a deadly bus accident in Trishuli where at least 54 people went missing after two passenger buses were swept away by a mudslide into the rain-swollen Trishuli River near Simaltal. Nepal police, Armed Police Force and Nepal Army personnel worked tirelessly to find the bus in the swelling river and continuous rainfall but the effort went into vain after the search had no result. Had the rescuers or finders had better technology, the last data recorded, the search would have been more predictive. The timely SOS would have been sent early. This is just an example of a larger problem that occurs in day to day life in a world moving so fast. There also are many instances where the drivers aren’t alerted, the state at which they are driving compared to a probabilistic scenario can be dangerous, if we find those pattern and pre-alert the driver , the accident case can be reduced and possibly save life of thousands travelling. This is why we chose “EDR-Based Intelligent Vehicle Accident Prediction with Multisensor Fusion and Edge Summarization”


\section{Problem Statement}
Existing systems for alerting or logging rely on single sensors or manual reporting, which makes them prone to false alarming and missed detection. The existing system depends on cloud based processing which in case of critical situations is very unrealistic and illogical because the delay in the immediate response required case often lead to major casualties. Safety, risk management and reliability are our main concerns overall. The lack of multisensor Fusion, no edge ML, poor logging are often not considered till date, even if those projects are never brought into a single module that can solve a larger problem scenario. By trying to integrate all the aforementioned problem’s solutions into a single module  we have tried to solve the existing problem. 

\section{Objectives}

The objectives of this project are as follows:
\begin{enumerate}
    \item To detect vehicle accidents or predict possible accident scenarios using Multi-Sensor Fusion and a Machine Learning model (Random Forest) implemented on a microcontroller using a TinyML approach, and to alert the driver or concerned authorities in real time.
    
    \item To log possible accidental parameters or critical spikes in sensor data during an accident or a potential accident scenario for further analysis.
\end{enumerate}




\section{Scope and Limitations}


\subsection{Scope}
The intelligence, detection and logging mechanism are the key features included in our project. We have used technologies like Multisensor Fusion, Edge summarization and Machine Learning model implemented on an ESP32 microcontroller using the TinyML approach. The system performs real-time data acquisition, local processing, alert generation and logging of critical sensor parameters during accident and pre-accident scenarios. The scope of this work is limited to small-scale prototype implementation and controlled testing conditions. The system does not include full-scale vehicle integration, cloud-based analytics, or legal emergency response systems.

\subsection{Limitations}
The project has a lot of scope, but due to time constraints and limited resources, some major components cannot be implemented in this phase. The addition of a camera module and accident detection through image classification using machine learning cannot be implemented yet. Limited resources also restrict us to using simple models rather than more complex ones at this stage of the project. In the future, with adequate time, computational power, and resources, advanced machine learning models along with camera-based image classification can be integrated to enhance accident detection accuracy.
