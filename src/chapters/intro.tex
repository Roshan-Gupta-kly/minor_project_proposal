\chapter{INTRODUCTION}

\noindent\textbf{CHAPTER GUIDE:} Purpose: set context, define the problem, and state objectives + scope. Must include: Background, Problem statement, Objectives, Scope and limitations. How to write: mention domain + current approaches; quantify gaps with metrics. LaTeX: Cite with cite key, reference figures and tables. General rules: formal/technical style; cite non-trivial claims; reference figures/tables in text.

\section{Background}

% Test abbreviations and symbols
This project uses \gls{ga} and \gls{ml} techniques \cite{deb2002nsga}. We also implement \gls{rl} with \gls{api} integration. The \gls{cpu} and \gls{gpu} work together for processing.

Research shows optimization methods are crucial in modern computing \cite{smith2020}. Following IEEE guidelines \cite{ieee2022author, ieee2021editorial}, we implement state-of-the-art algorithms.

The learning rate \gls{symb:alpha} is crucial, and we use discount factor \gls{symb:gamma}. Standard deviation \gls{symb:sigma} helps measure variance with mean \gls{symb:mu}.

\section{Problem Statement}

\section{Objectives}
Objevtives should be SMART: Specific, Measurable, Achievable, Relevant, Time-bound. List primary and secondary objectives clearly.



\section{Scope and Limitations}
\subsection{Scope}

\subsection{Limitations}