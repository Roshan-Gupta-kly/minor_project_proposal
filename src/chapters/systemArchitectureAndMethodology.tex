\chapter{SYSTEM ARCHITECTURE AND METHODOLOGY}

\section{Proposed System Architecture}
\subsection{Proposed System Block Diagram}
\begin{figure}[H]
    \centering
    \includegraphics[width=15cm]{src/images/figures/block.jpg}
    \caption{Block Diagram of Proposed System Architecture}
    \label{fig:Block Diagram}
\end{figure}

\section{Working Principle}
The proposed system uses a combination of multisensor fusion and Tiny\gls{ml}-based event classification (Random Forest) to monitor vehicles in pre-event and post-event analysis in real time, responding to potential accidents and generating alerts when necessary. The system is built around two microcontrollers, with one responsible for multisensor fusion and the other for verifying accident parameters via the \gls{ml} model for prediction or alert.

The \gls{esp32s3} Module I acquires continuous data from multiple sensors including \gls{acc}, \gls{gy}, \gls{ms}, \gls{gps}, \gls{fs}, and \gls{cs}. These sensors collectively provide comprehensive information regarding the motion, orientation, location, and critical environment conditions of a vehicle. Significant deviations in sensor data are forwarded to the second module using decision inference. 

A multisensor fusion algorithm based on the Kalman Filter is employed, integrating readings from all sensors to produce robust and reliable state estimation while minimizing noise and transient errors. Features of potential abnormal events are extracted from raw sensor readings, including acceleration magnitude, orientation angles, sudden speed changes, and activation of crash or flame sensors, forming the input for \gls{ml}-based classification.

The \gls{esp32s3} Module II performs the \gls{ml} inference to classify events as normal, warning, or accident. For normal events, the system continues monitoring without intervention. For warning events, the system triggers a local alert such as a buzzer. When an accident is detected, the system activates the \gls{gsm} module to send a \gls{sos} message to predefined emergency contacts. A brief observation window ensures persistent abnormal conditions are confirmed before sending \gls{sos} signals, reducing false alarms.

Overall, the system integrates real-time sensor monitoring, robust data fusion, and lightweight \gls{ml} inference to provide a reliable, fast, and energy-efficient solution for intelligent vehicle accident detection. Preventive warnings and emergency communication improve vehicle safety while maintaining low computational requirements suitable for microcontroller-based embedded platforms.

\section{Kalman Filter Algorithm}
The Kalman Filter is an optimal recursive estimation algorithm that simplifies the computational process of data fusion by introducing assumptions regarding system dynamics, state evolution, and the statistical properties of noise and estimation errors \cite{kalman1960filtering}. It provides an efficient solution to the linear quadratic estimation problem for systems affected by stochastic disturbances.

The Kalman Filter formulates estimates of the internal state of a linear dynamic system by processing a sequence of noisy measurements. The system is assumed to be disturbed by zero-mean Gaussian white noise, and the filter recursively updates the state estimates as new measurements become available \cite{montella2011kalmanreview}. Its predictor--corrector structure can be divided into two distinct phases:
\begin{itemize}
    \item Time Update (Prediction)
    \item Measurement Update (Correction)
\end{itemize}

\subsection{Assumptions}
The Kalman Filter operates under the following assumptions \cite{kalman1960filtering}:
\begin{enumerate}
    \item Linear system dynamics:
    \begin{equation}
        \mathbf{s}_t = \mathbf{A}\mathbf{s}_{t-1} + \mathbf{B}\mathbf{u}_t + \mathbf{w}_t
    \end{equation}
    \item Linear measurement model:
    \begin{equation}
        \mathbf{z}_t = \mathbf{H}\mathbf{s}_t + \mathbf{v}_t
    \end{equation}
    \item Process and measurement noises are mutually independent.
\end{enumerate}

\subsection{Time Update (Prediction Phase)}
\subsubsection*{State Prediction}
\begin{equation}
    \hat{\mathbf{s}}_{t|t-1} = \mathbf{A}\hat{\mathbf{s}}_{t-1|t-1} + \mathbf{B}\mathbf{u}_t
\end{equation}

\subsubsection*{Covariance Prediction}
\begin{equation}
    \mathbf{P}_{t|t-1} = \mathbf{A}\mathbf{P}_{t-1|t-1}\mathbf{A}^T + \mathbf{Q}
\end{equation}

\subsection{Measurement Update (Correction Phase)}
\subsubsection*{Kalman Gain}
\begin{equation}
    \mathbf{K}_t = \mathbf{P}_{t|t-1}\mathbf{H}^T
    \left(
    \mathbf{H}\mathbf{P}_{t|t-1}\mathbf{H}^T + \mathbf{R}
    \right)^{-1}
\end{equation}

\subsubsection*{Innovation (Measurement Residual)}
\begin{equation}
    \mathbf{y}_t = \mathbf{z}_t - \mathbf{H}\hat{\mathbf{s}}_{t|t-1}
\end{equation}

\subsubsection*{State Update}
\begin{equation}
    \hat{\mathbf{s}}_{t|t} =
    \hat{\mathbf{s}}_{t|t-1} + \mathbf{K}_t \mathbf{y}_t
\end{equation}

\subsubsection*{Covariance Update}
\begin{equation}
    \mathbf{P}_{t|t} =
    \left(
    \mathbf{I} - \mathbf{K}_t \mathbf{H}
    \right)
    \mathbf{P}_{t|t-1}
\end{equation}

\section{Random Forest Machine Learning Model}
A random forest is an ensemble of tree-structured classifiers that vote for the most popular class at an input vector. It performs Tiny\gls{ml} inference to classify events as normal, warning, or accident \cite{breiman2001random}.

\subsection{Key Features of Random Forests}
\begin{itemize}
    \item High accuracy and robustness to outliers and noise~\cite{breiman2001random}.
    \item Relatively fast training and inference compared to bagging or boosting~\cite{breiman2001random}.
    \item Provides useful internal estimates of error, variable importance, and correlation~\cite{breiman2001random}.
    \item Easily parallelizable and suitable for microcontroller deployment~\cite{scribd2025}.
\end{itemize}

\begin{figure}[H]
    \centering
    \includegraphics[width=15cm]{src/images/figures/flowchart.jpg}
    \caption{Flowchart of Proposed System Architecture}
    \label{fig:flowchart}
\end{figure}
