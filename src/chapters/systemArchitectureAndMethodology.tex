\chapter{SYSTEM ARCHITECTURE AND METHODOLOGY}

Present overall system design, component interactions, and methods to achieve objectives.
Enable readers to understand system structure before implementation details.

\textbf{Customize the sections below based on your project's nature (hardware, software, ML, etc.). and also include relevant diagrams. }

\section{System Overview}
High-level description of system with block diagrams showing components, data flow, and interactions.

\section{Problem Formulation}
If relevant, mathematically define the problem your system addresses. Include equations, constraints, and assumptions.



Note: On second use, abbreviations show short form: \gls{ga}, \gls{ml}, \gls{rl}.

% Test algorithm
\begin{algorithm}
    \caption{Genetic Algorithm for Optimization}
    \label{alg:ga_test}
    \begin{algorithmic}[1]
        \State Initialize population $P$ with random solutions
        \State Evaluate fitness for each individual in $P$
        \While{termination condition not met}
        \State Select parents using tournament selection
        \State Apply crossover to create offspring
        \State Apply mutation to offspring
        \State Evaluate fitness of offspring
        \State Replace worst individuals with offspring
        \EndWhile
        \State \Return best solution from $P$
    \end{algorithmic}
\end{algorithm}

See \cref{alg:ga_test} for the genetic algorithm pseudocode.


% Test algorithm
\begin{algorithm}
\caption{Genetic Algorithm for Optimization}
\label{alg:ga_test2}
\begin{algorithmic}[1]
\State Initialize population $P$ with random solutions
\State Evaluate fitness for each individual in $P$
\While{termination condition not met}
    \State Select parents using tournament selection
    \State Apply crossover to create offspring
    \State Apply mutation to offspring
    \State Evaluate fitness of offspring
    \State Replace worst individuals with offspring
\EndWhile
\State \Return best solution from $P$
\end{algorithmic}
\end{algorithm}

See \cref{alg:ga_test2} for the genetic algorithm pseudocode.



\section{Hardware/Software/Data Architecture}
HARDWARE: Processing unit selection with justification (power, memory, cost). Sensor/actuator subsystems.
ML/SOFTWARE: End-to-end pipeline diagram. Data sources, storage, preprocessing. Model architecture.
Justify technology stack choices.

\section{Communication/Model Architecture}
HARDWARE: Internal (I2C, SPI, UART) and external (WiFi, Bluetooth) communication with protocol diagrams.
ML: Training methodology (loss functions, optimization, regularization), validation strategy.
SOFTWARE: Architectural style (microservices, layered), major components with UML/sequence diagrams.

\section{Power Management / Deployment Architecture}
HARDWARE: Power sources, consumption analysis, efficiency strategies, battery life calculation.
ML: Serving infrastructure, API design, monitoring, model versioning.
SOFTWARE: Deployment platform, scalability, CI/CD pipeline.

HARDWARE: Power sources, consumption analysis, efficiency strategies, battery life calculation.
ML: Serving infrastructure, API design, monitoring, model versioning.

\section{Methodology}
% Development approach: requirements analysis, component selection, simulation/prototyping, testing strategy.
% Include relevant diagrams: block diagrams, schematics, flowcharts, ER diagrams, API specs.