\chapter{IMPLEMENTATION DETAILS}

\section{Hardware Implementation}

\subsection{Accelerometer and Gyroscope Sensors}
An \gls{acc} and \gls{gy} together form an \gls{imu} used to measure linear acceleration and angular velocity of the vehicle. These sensors continuously monitor dynamic motions such as acceleration, braking, skidding, tilting, sharp turns, rollover, and collision forces.\\

The accelerometer and gyroscope together measure both linear and rotational motion of the vehicle, enabling accurate detection of sudden impacts and abnormal movements. They provide continuous, low-noise motion data that is well suited for real-time processing while maintaining low power consumption and minimal computational overhead. Due to their compact size, cost-effectiveness, and ease of integration with microcontrollers, these sensors are widely used in embedded safety systems. Furthermore, the combined motion data from the \gls{acc} and \gls{gy} is highly compatible with both rule-based and \gls{ml}-based accident detection systems. This integrated sensing capability allows effective feature extraction for identifying unsafe driving behavior, loss of vehicle stability, and crash events, making the \gls{imu} a key component in accident prediction and vehicle safety applications.
\\
The MPU6050 is a 6-axis MEMS \gls{imu} that integrates a 3-axis \gls{acc} and a 3-axis \gls{gy} along with an internal Digital Motion Processor (DMP). It outputs motion data digitally via the I\textsuperscript{2}C interface. By combining acceleration and rotational data, it effectively detects sudden motion changes and crash occurrences.\\

\noindent\textbf{I\textsuperscript{2}C Interface}

In today’s technological world, many applications such as auto mobiles, laptops, embedded devices and other peripherals need to connect multiple devices. We also need to make them able to communicate with each other and this requirement gives rise to the need for any medium or channel which can act as a bridge between these peripherals to share data or information. There are many communication protocols for this purpose. One of them is Inter-Integrated circuit (I2C or IIC) protocol. I2C has two-wires named SDA (serial data line) and SCL (serial clock line), the bidirectional serial bus that provides a simple and efficient communication between devices. It is multi-master and multi-slave protocol but single-master and the multi-slave combination are mostly used. Master is a device which initiates transactions and generates a clock signal. The slave is a device which is being addressed by the master.\cite{pandey2018reviewi2c} 


\begin{figure}[H]
    \centering
    \includegraphics[width=5cm]{src/images/figures/mpu6050.jpg}
    \caption{MPU6050 Accelerometer and Gyroscope Module}
    \label{fig:MPU6050}
\end{figure}

%------------------------------------------------------------

\subsection{Magnetometer Sensor}
A \gls{ms} measures the strength and direction of a magnetic field, typically the Earth’s magnetic field, to determine vehicle orientation and heading.\\

The magnetometer provides an absolute heading reference by measuring the Earth’s magnetic field, which is essential for determining the vehicle’s orientation. It improves the long-term stability of motion estimation by compensating for drift errors that accumulate in accelerometer and gyroscope data. The sensor operates with low power consumption while delivering high-resolution digital output, making it suitable for continuous real-time applications. Additionally, magnetometers are cost-effective and highly compatible with \gls{ml}-based systems, as the heading and magnetic field features can be effectively utilized for orientation-aware accident prediction and vehicle safety analysis.
\\
The HMC5883L is a 3-axis MEMS \gls{ms} that uses magnetoresistive sensing elements to measure magnetic field components along the X, Y, and Z axes and provides digital output through the I\textsuperscript{2}C interface.

\begin{figure}[H]
    \centering
    \includegraphics[width=5cm]{src/images/figures/HMC5883L.jpg}
    \caption{HMC5883L Magnetometer Module}
    \label{fig:HMC5883L}
\end{figure}

%------------------------------------------------------------

\subsection{GPS Sensor}
A \gls{gps} sensor determines the absolute geographical position of a vehicle by receiving signals from multiple satellites. It provides latitude, longitude, speed, altitude, and timestamp information essential for vehicle tracking and accident localization.
\\
The GPS module provides accurate real-time latitude and longitude information, enabling precise tracking of the vehicle’s geographical location. It also measures vehicle speed and heading, which are essential parameters for understanding motion patterns and driving behavior. In the event of an accident, the GPS module enables exact location reporting, allowing emergency services to respond quickly and efficiently. The module operates with low power consumption, making it suitable for continuous tracking applications. Additionally, the stable and reliable data output from the GPS module is well suited for \gls{ml}-based processing, supporting location-aware accident prediction and post-accident analysis.

\begin{figure}[H]
    \centering
    \includegraphics[width=5cm]{src/images/figures/gps.jpg}
    \caption{GPS NEO-6M Module}
    \label{fig:NEO-6M}
\end{figure}

%------------------------------------------------------------

\subsection{Flame Sensor}
A \gls{fs} detects the presence of fire by sensing infrared radiation emitted during combustion. It is critical for identifying fire hazards following vehicle collisions.\\

The flame sensor enables early detection of fire by identifying infrared radiation emitted from flames, which is critical for enhancing vehicle safety during post-crash or fault conditions. It offers a fast response time, allowing the system to react promptly to fire-related hazards. The sensor features a simple interface that can be easily integrated with microcontrollers, while maintaining low power consumption and cost. Additionally, the output from the flame sensor is suitable for both rule-based and \gls{ml}-based decision systems, supporting reliable fire detection and emergency response in accident prediction and vehicle safety applications.

\begin{figure}[H]
    \centering
    \includegraphics[width=5cm]{src/images/figures/flame.jpg}
    \caption{Flame Sensor Module}
    \label{fig:flame}
\end{figure}

%------------------------------------------------------------

\subsection{Crash Sensor}
A \gls{cs}, also known as an impact sensor, detects sudden mechanical shocks or collision forces acting on the vehicle.
\\
The crash sensor enables immediate detection of collision events by sensing sudden impact forces acting on the vehicle. It provides high reliability in impact sensing, ensuring accurate identification of crash conditions with minimal false triggers. The sensor requires very low processing overhead, making it suitable for real-time embedded systems. Its fast response time allows rapid activation of emergency mechanisms such as alerts and \gls{sos} signaling. Additionally, the crash sensor is cost-effective and durable, and it provides strong support for decision inference when combined with multisensor fusion and \gls{ml}-based accident detection frameworks.

\begin{figure}[H]
    \centering
    \includegraphics[width=5cm]{src/images/figures/crash.jpg}
    \caption{Crash Sensor Module}
    \label{fig:crash}
\end{figure}

%------------------------------------------------------------

\subsection{GSM Module}
A \gls{gsm} module enables embedded systems to send and receive data over cellular networks using a \gls{sim}. It allows communication with predefined contacts such as emergency services and family members.\\

The GSM module communicates with the controller via \gls{uart} using \gls{at} commands. Once a \gls{sim} is inserted, the module registers with the nearest cellular base station. During emergency events such as crashes or fire detection, the module automatically transmits SMS alerts to predefined mobile numbers.\\

The GSM module enables long-distance communication without dependency on internet connectivity, making it suitable for vehicle safety applications in diverse environments. It supports real-time transmission of emergency alerts through SMS, ensuring timely notification to predefined contacts or emergency services. The module operates reliably in rural and remote areas where internet access may be limited. Its simple interface allows easy integration with microcontrollers, while maintaining low power consumption. Additionally, GSM modules are cost-effective and widely supported, making them a practical choice for reliable emergency communication in accident detection and response systems.

\begin{figure}[H]
    \centering
    \includegraphics[width=5cm]{src/images/figures/sim800l.png}
    \caption{GSM SIM800L Module}
    \label{fig:GSM}
\end{figure}

%------------------------------------------------------------

\subsection{ESP32-S3 Microcontroller}
The \gls{esp32s3} is a high-performance, low-power microcontroller developed by Espressif Systems for advanced embedded and IoT applications. It integrates wireless connectivity and AI acceleration features, making it suitable for real-time sensing and edge intelligence.\\


The \gls{esp32s3} is based on a dual-core Xtensa\textsuperscript{\textregistered} 32-bit LX7 processor operating up to 240 MHz. It includes built-in Wi-Fi and \gls{ble}, along with rich peripheral support such as \gls{spi}, I\textsuperscript{2}C, \gls{uart}, \gls{adc}, \gls{pwm}, and \gls{gpio}s. Support for external flash and PSRAM enables efficient handling of sensor data and \gls{ml} models.
\\

The ESP32-S3 offers high processing capability, making it suitable for handling real-time sensor data acquisition and decision-making tasks. It is optimized for \gls{ml} inference, enabling efficient execution of lightweight machine learning models directly on the device. The microcontroller supports simultaneous interfacing with multiple sensors, which is essential for multisensor fusion in vehicle safety systems. Additionally, the ESP32-S3 operates with low power consumption, making it well suited for continuous monitoring applications in embedded and edge-based accident detection systems.
\\

The \gls{esp32s3} is specifically designed for edge \gls{ml} applications. Its LX7 processor supports vector instructions that accelerate mathematical operations used in \gls{ml} inference. Frameworks such as TensorFlow Lite for Microcontrollers and ESP-DSP enable deployment of trained models directly on the device. In this project, the \gls{esp32s3} performs real-time sensor fusion, feature extraction, and decision inference locally, reducing latency and eliminating cloud dependency. This makes it ideal for safety-critical applications such as crash and fire detection.

\begin{figure}[H]
    \centering
    \includegraphics[width=5cm]{src/images/figures/esp32.jpg}
    \caption{ESP32-S3 Module}
    \label{fig:ESP32}
\end{figure}


\begin{figure}[H]
    \centering
    \includegraphics[width=5cm]{src/images/figures/esp32_config.jpg}
    \caption{ESP32-S3 Configuration}
    \label{fig:ESP32_config}
\end{figure} 

The ESP32-S3 is a highly integrated system-on-chip (SoC) designed for embedded and IoT applications, offering a versatile pin configuration to support a wide range of peripherals and interfaces. The device operates on a 3.3\,V power supply, provided through multiple power pins such as VDD3P3, VDD3P3\_CPU, and VDD\_SPI, with a common ground reference to ensure stable operation. The CHIP\_PU pin controls the enable and reset functionality of the chip, allowing controlled startup and power management. Clocking is achieved through the XTAL\_P and XTAL\_N pins, which connect to an external high-frequency crystal oscillator, while optional low-power timing support is available through the XTAL\_32K\_P and XTAL\_32K\_N pins.

The ESP32-S3 provides a large number of general-purpose input/output (GPIO) pins that can be flexibly configured for digital input and output, analog-to-digital conversion, pulse-width modulation, and various communication protocols including UART, SPI, and I$^2$C through an internal GPIO matrix. Native USB 2.0 full-speed support is integrated into the chip and is enabled through the USB D\texttt{+} and USB D\texttt{--} pins, allowing direct USB communication without the need for an external USB-to-UART interface. External non-volatile memory and optional PSRAM are interfaced through dedicated SPI pins, which are typically reserved for memory operations. Debugging and in-circuit testing are supported through JTAG interface pins, while wireless communication is facilitated via the \texttt{LNA\_IN} radio-frequency input connected to the antenna circuitry.




\section{Software Implementation}
The software implementation involves programming the \gls{esp32s3} microcontroller using the Arduino \gls{ide} and \gls{espidf}. The code integrates sensor data acquisition, preprocessing, feature extraction, and decision-making algorithms for accident and fire detection. The \gls{ml} model is trained offline using Python-based libraries such as NumPy and Pandas, and then converted to a format compatible with the \gls{esp32s3} for deployment. The system continuously monitors sensor inputs, processes the data in real-time, and triggers alerts via the GSM module when hazardous events are detected.\\

\subsection{Arduino IDE}
Arduino IDE is a widely used software platform for writing, compiling, and uploading programs to microcontroller boards such as Arduino and NodeMCU. It provides a simple and intuitive environment that allows developers to efficiently develop embedded applications. The IDE includes a built-in code editor with syntax highlighting, a compiler, and tools for uploading firmware to hardware devices. It also supports a large number of libraries, enabling easy integration of sensors, communication modules, and peripheral devices. Arduino IDE is compatible with multiple operating systems, making it accessible and convenient for developers.\\

\subsection{TensorFlow}
TensorFlow is an open-source software framework developed by Google for numerical computation and data processing. It provides a flexible platform for handling large datasets, performing mathematical operations, and building computational models. TensorFlow offers a rich set of libraries and tools that help developers process and analyze data efficiently. In this project, it is used as a supportive software tool for data handling and model development during the system design and evaluation phase.\\

\subsection{VS code IDE}
VS Code IDE is a powerful and versatile code editor developed by Microsoft. It provides a rich set of features for writing, debugging, and managing code across various programming languages. VS Code supports extensions that enhance its functionality, including support for Python development, which is essential for data analysis and \gls{ml} model development in this project. The IDE offers integrated terminal access, version control integration, and customizable settings, making it a preferred choice for developers working on complex software projects.\\

\subsection{Jupyter Notebook}
Jupyter Notebook is an open-source web application that allows users to create and share documents containing live code, equations, visualizations, and narrative text. It is widely used for data analysis, scientific computing, and machine learning tasks. In this project, Jupyter Notebook serves as a supportive tool for exploratory data analysis, feature engineering, and model training. It provides an interactive environment where developers can write and execute Python code, visualize data using libraries like Matplotlib and Seaborn, and document their findings in a structured format.\\

\section{Datasets}
The proposed system will be using a sensor-based dataset to train a TinyML model  relying  on the accelerometer, gyroscope, GPS data. The dataset we use will not be the data under crash condition as these kinds of data are difficult to find on internet and difficult to collect ourselves. The better option would be to use the dataset under normal driving  conditions. The data was found on the website “Kaggle”. This data contains the motion sensor-based data collected during various driving conditions including normal vehicle movement sharp turns, road bumps  which are essential for the accurate prediction of accident.\cite{kaggle_dataset2025} 
\\
 
As given on the website, the devices were fitted on the vehicle as shown in the figure below. Data of many different components including accelerometer, gyroscope, GPS, camera was collected. We plan to use only the data of accelerometers, gyroscope and GPS. \cite{kaggle_notebook2025} The data were produced in three different vehicles, with three different drivers, in three different environments in which there are three different surface types, in addition to variations in conservation state and presence of obstacles and anomalies, such as speed 20 bumps and potholes. The data was collected from three different types of roads i.e. Asphalt Road, Dirt Road and Cobblestone Road. The dataset contains the data of good roads, bad roads, and regular roads. Speed bumps in Cobblestone and Asphalt roads are also included in the dataset. It is also inclusive of driving events, such as lane change, braking, skidding, aquaplaning, turning right or left etc. \\

\begin{figure}[H]
    \centering
    \includegraphics[width=15cm]{src/images/figures/dataset.jpg}
    \caption{Datasets}
    \label{fig:dataset}
\end{figure}
