\chapter{LITERATURE REVIEW}


Amin et al.~\cite{amin2016accident} proposed an accident detection system that utilizes deceleration data from low-cost Micro Electro Mechanical Systems (MEMS) based Inertial Measurement Units (IMU). Their research demonstrates that the system can accurately detect collisions and maintain location tracking during GPS outages, successfully transmitting critical information to a base station.

Furthering the integration of connectivity, Kumar et al.~\cite{kumar2021iot} presented an IoT-based automotive accident detection and classification (ADC) system. By fusing smartphone-built-in sensors with connected external sensors, this system not only detects accidents but also classifies the type of collision. This detailed reporting improves the efficacy of emergency medical services (EMS), fire departments, and towing services by allowing for more precise resource planning.

Regarding data management, Yao and Atkins~\cite{yao2021smart} introduced the "Smart Black Box" (SBB), which enhances traditional low-bandwidth logging with value-driven, high-bandwidth data capture. The SBB utilizes a deterministic Mealy machine to cache short-term data buffers based on similarity and information value. By formulating compression as a constrained multi-objective optimization problem, the system prioritizes high-value recordings and discard redundant data to optimize finite storage.

Predivtive modeling is addressed by Yang et al.~\cite{yang2023prediction}, who employed a Random Forest algorithm to predict the severity of traffic accidents. By incorporating four key characteristics such as location, accident form, road information, and driving speed into thier model, they achieved high performance in data classification, resulting in a more efficient and accurate predictive framework.

This project proposes an integrated vehicle safety system that combines accident prediction and detection with a robust data logging framework for forensic investigation. By utilizing a Random Forest algorithm, the system can identify potential risks and detect vehicle accident in real-time, capturing critical pre-crash and post-crash telemetry. To ensure the accuracy of these observations, the system employs a linear Kalman filter to fuse data from the accelerometer, gyroscope, and GPS, effectively minimizing sensor noise and measurement errors.

Furthermore, the system addresses data management challenges through an event summarization technique. This method prioritizes the storage of relevant incident data, successfully reducing overall storage capacity requirements by half without compromising the integrity of the information needed for future investigations. Compromising the integrity of the information needed for future investigations.
