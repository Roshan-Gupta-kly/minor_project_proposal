

\chapter*{ABSTRACT}
\phantomsection
\addcontentsline{toc}{chapter}{ABSTRACT}

\noindent\textbf{ABSTRACT GUIDE:} 150-300 words, include: Problem context, Approach, Key results, Impact. Avoid citations, unexpanded acronyms, detailed implementation.

\bigskip

This report presents the development and evaluation of an intelligent, adaptive scheduling system for the University Course Timetabling Problem (UCTP) using a hybrid approach combining NSGA-II genetic algorithms with reinforcement learning-based hyper-heuristic control. UCTP is an NP-hard combinatorial optimization problem requiring the assignment of courses to time slots, rooms, and instructors while satisfying complex hard and soft constraints. The implemented system features a modular architecture with multi-objective optimization (minimizing hard constraint violations and soft constraint penalties separately), CPU-parallelized fitness evaluation, and a trained PPO agent that adaptively selects from 20 repair heuristics based on a 39-dimensional population state vector. Experiments on an institutional-scale dataset from Tribhuvan University (444 courses generating 668 sessions, 37 student groups, 181 instructors, 67 rooms) demonstrate the system's ability to generate high-quality timetables through progressive experimental modes. The hybrid GA-RL architecture successfully balances exploration and exploitation, with the reinforcement learning controller learning effective phase-dependent strategies for heuristic selection. Performance analysis reveals significant improvements in constraint satisfaction rates and Pareto front quality compared to baseline NSGA-II, validating the hyper-heuristic approach for large-scale educational timetabling.

\vspace{1em}
\noindent\textit{\textbf{Keywords}}---Combinatorial optimization, constraint satisfaction, genetic algorithm, hyper-heuristics, metaheuristic search, NP-hard problems, reinforcement learning, university course timetabling problem.